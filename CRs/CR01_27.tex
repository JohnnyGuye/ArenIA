\documentclass[10pt]{article}
\usepackage[utf8]{inputenc} 

\begin{document}

\part*{{Réunion du 27/01/2016}}

\renewcommand{\contentsname}{Sommaire}
\tableofcontents





\newpage

\section{Les Robots}

\subsection{Description générale}

Un robot a plusieurs caractéristiques : 

\begin{itemize}
\item PV (Points de Vie)
\item Energie
\item Armure
\item Vitesse de déplacement
\item Champ de vision (qui est aussi sa fenêtre de tir)
\item Compétences exécutables (dont tirer)

\end{itemize}

A chaque tour, un robot prend une décision, puis effectue des actions.



\subsection{Paramètres pris en compte pour la décision}

Les paramètres que peut prendre en compte un robot sont :
\begin{itemize}

\item "GetObjet" : son environnement.

Il peut à la fois détecter les obstacles, ses alliés et ses ennemis, et les caractéristiques de ces derniers (PV, orientation, direction).

\item Ses paramètres internes : PV, orientation, vitesse, position relative à la carte. 

\end{itemize}



\subsection{Actions réalisables}

Toutes les actions sont compatibles entre elles (sauf utiliser plusieurs compétences en même temps et effectuer des actions contradictoires [ex : tourner la tourelle à gauche et à droite]). Dans le cas où on essaye de lancer deux actions contradictoires, le dernier ordre prévaut.

\paragraph{}
Un robot peut :

\begin{itemize}
\item Utiliser une compétence (le tir en fait partie)
\item Tourner sa tourelle
\item Tourner ses roues
\item Avancer
\end{itemize}



\newpage

\section{Les Parties}

On lance une partie à partir d'une map et d'équipes de robots.

A chaque tour, tous les robots prennent leurs décisions \textbf{en même temps} (pour ne pas en favoriser un), on détecte/résout les problèmes de collisions, et les robots effectuent leurs actions.

On donne un nombre limite de tours après lequel le match se transforme en mort subite.



\section{Roadmap}

\begin{itemize}
\item Prototype fonctionnel ( lancement en ligne de commande "ArenIA Robot1 Robot2" , fenêtre où se déroule le match)
\item Possibilité d'avoir des équipes
\item GUI
\item 0bstacles
\item Objectifs (donc plusieurs maps)
\item Mode multijoueur
\item Tutoriel
\item Mode histoire
\paragraph{}
\item Déploiement, DLCs, tons of money, meeting John Cena and world domination.
\end{itemize}




\section{Répartition du travail}

\begin{tabular}{| c | c |}

\hline
Tâche & Personne(s) \\
\hline
Son & Clément, Johnny \\
\hline
Robot & Alp, Simon \\
\hline
GFX & Johnny, Samory \\
\hline
Moteur physique & Clément \\
\hline
IA & Clément \\
\hline

\end{tabular}



\section{UML}
cf UML de Johnny sur draw.io

\end{document}